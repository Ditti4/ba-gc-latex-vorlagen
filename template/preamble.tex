% Präambel, um alle Dokumentendefinitionen durchzuführen

% Vorgabe der BA-Glauchau: Schriftgroesse: 12pt, A4. Blocksatz
\documentclass[fontsize=12pt, paper=a4, utf8]{scrartcl}

% Vorgabe der BA-Glauchau: Raender links 2.5cm, rechts 2.5cm, oben 2cm, unten 2cm
\usepackage[a4paper, left=2.5cm, right=2.5cm, top=2cm, bottom=2cm, bindingoffset=5mm, footskip=20pt, includeheadfoot]{geometry}
\usepackage{lmodern}
\usepackage[ngerman]{babel}
\usepackage[utf8]{inputenc}
\usepackage[T1]{fontenc}
% Rand ist Rand und kein Gartentor.
\usepackage{microtype}
%%%%%%%%%%%%%%%%%%%%%%%%%%%%%%%%%%%%%%%%%%%%%%%%%%%%%%%%%%%%
% Arbeitslaptop-Hack
\renewcommand{\sfdefault}{phv}
\renewcommand{\rmdefault}{phv}
%%%%%%%%%%%%%%%%%%%%%%%%%%%%%%%%%%%%%%%%%%%%%%%%%%%%%%%%%%%%
% korrekter wäre
%\usepackage{uarial}
%%%%%%%%%%%%%%%%%%%%%%%%%%%%%%%%%%%%%%%%%%%%%%%%%%%%%%%%%%%%
% benötigt, um gesamte Tabellenspalte z.B. fett darzustellen
\usepackage{array}
% yay, fancy stuff (Header und Footer)
\usepackage{fancyhdr}
% Lorem Ipsum is love, Lorem Ipsum is life
\usepackage{lipsum}
\usepackage{setspace}
\usepackage{forloop}
% sollte irgendwann mal irgendjemand \textsubscript brauchen
\usepackage{fixltx2e}
% Paketname sprich für sich
\usepackage{ifthen}
% colorlinks = true, damit kein hässlicher Rahmen um die Links herum entsteht
\usepackage[linkcolor=black, urlcolor=black, citecolor=black, colorlinks=true]{hyperref}
% auf Rechnern mit Biber bitte den backend-Parameter entfernen, ist nur wieder
% ein kleiner Hack für meinen Arbeitslaptop
\usepackage[backend=bibtex]{biblatex}
% benötigt für sectionbreak
\usepackage{titlesec}
% Inline-Code-Snippets
\usepackage{listings}
% "In Farbe und bunt."
\usepackage{color}
% ToDo-Stuff, recht offensichtlich
\usepackage{todonotes}
% Zum Einbinden des Themenvorschlags
\usepackage{pdfpages}
% Paket für ein einfaches Abkürzungsverzeichnis
\usepackage{nomencl}

% fortlaufende Fußnotennummerierung unabhängig vom Kapitel
% nur bei scrreprt/report nötig
% \counterwithout{footnote}{section}

% neue Seite bei jedem Kapitel
% \chapter gibt es bei article nicht, deswegen sectionbreak
\newcommand{\sectionbreak}{\clearpage}

% wiederholt Zeichenketten gewünscht oft
% wird z.B. in der ehrenwörtlichen Erklärung verwendet
\newcounter{charcounter}
\newcommand{\rpt}[2]{
    \forloop{charcounter}{0}{\value{charcounter} < #1}{#2}
}

% Standardschriftartenfamilie auf Sans Serif festlegen
\renewcommand{\familydefault}{\sfdefault}

% kein Absatzeinschub
\setlength{\parindent}{0mm}

% Befehle für das Titelblatt einbinden
\newboolean{inhaltsverzeichnis}
\newboolean{abkuerzungsverzeichnis}
\newboolean{tabellenverzeichnis}
\newboolean{abbildungsverzeichnis}
\newboolean{formelverzeichnis}
\newboolean{themenvorschlag}
\newboolean{literaturverzeichnis}

\newcommand{\settype}[1]{
	\newcommand{\reporttype}{#1}
}

\newcommand{\settopic}[1]{
	\newcommand{\reporttopic}{#1}
}

\newcommand{\setsubmissiondate}[1]{
	\newcommand{\submissiondate}{#1}
}

\newcommand{\setstudypath}[1]{
	\newcommand{\studypath}{#1}
}

\newcommand{\setstudygroup}[1]{
	\newcommand{\studygroup}{#1}
}

\newcommand{\setauthorone}[5]{
	\newcommand{\authoronename}{#1}
	\newcommand{\authoroneaddress}{#2}
	\newcommand{\authoronepostalcity}{#3}
	\newcommand{\authoronestudentnumber}{#4}
	\newcommand{\authoronestudybranch}{#5}
}

\newcommand{\setauthortwo}[5]{
	\newcommand{\authortwoname}{#1}
	\newcommand{\authortwoaddress}{#2}
	\newcommand{\authortwopostalcity}{#3}
	\newcommand{\authortwostudentnumber}{#4}
	\newcommand{\authortwostudybranch}{#5}
}

\newcommand{\setauthorthree}[5]{
	\newcommand{\authorthreename}{#1}
	\newcommand{\authorthreeaddress}{#2}
	\newcommand{\authorthreepostalcity}{#3}
	\newcommand{\authorthreestudentnumber}{#4}
	\newcommand{\authorthreestudybranch}{#5}
}

\newcommand{\setcompanyinfo}[3]{
	\newcommand{\companyname}{#1}
	\newcommand{\companyaddress}{#2}
	\newcommand{\companypostalcity}{#3}
}

\newcommand{\setcompanyreviewer}[1]{
	\newcommand{\companyreviewername}{#1}
}

\newcommand{\setunireviewer}[1]{
	\newcommand{\universityreviewername}{#1}
}

\newcounter{authorcount}


% Header- und Footer-Style festlegen
\fancypagestyle{report-page}{
    \fancyhf{}
    \fancyhead[R]{\rightmark}
    \renewcommand{\headrulewidth}{0.4pt}
    \fancyfoot[R]{Seite \thepage}
    \renewcommand{\footrulewidth}{0.4pt}
}

\fancypagestyle{plain}{
    \fancyhf{}
    \fancyhead[R]{\leftmark}
    \renewcommand{\headrulewidth}{0.4pt}
    \fancyfoot[R]{Seite \thepage}
    \renewcommand{\footrulewidth}{0.4pt}
}

% irgendwas mit den Fußnoten … alles viel zu lange her
\setlength{\footnotesep}{10pt}
% kein Absatzeinschub
\setlength{\parindent}{0pt}

% Standardschriftartenfamilie auf Sans Serif festlegen
\renewcommand{\familydefault}{\sfdefault}

% in den Verzeichnissen die Punkte zum Auffüllen bis zur Seitenzahl direkt
% hintereinander ausgeben (ohne echten Abstand)
\renewcommand{\cftdotsep}{0.1}
\renewcommand{\cftsecleader}{\cftdotfill{\cftdotsep}}

% Inhaltsverzeichnis richtig einrücken
\cftsetindents{section}{0em}{55pt}
\cftsetindents{subsection}{0em}{55pt}
\cftsetindents{subsubsection}{0em}{55pt}
\cftsetindents{paragraph}{0em}{55pt}
% um ganz den Richtlinien zu entsprechen, muss der Abstand zwischen Kapiteln
% genauso groß sein wie der zwischen Unterkapiteln, das sieht jedoch schrecklich
% und sehr unübersichtlich aus
% wer sich das dennoch freiwillig antun möchte, darf bei der folgenden Zeile
% gern das % entfernen
%\setlength{\cftbeforesecskip}{0pt}

% Abbildungsverzeichnis richtig einrücken und fett drucken
\setlength{\cftfigindent}{0pt}
% etwas großzügiger bemessen, damit auch Abbildungen mit zweistelligen Nummern
% noch ins Verzeichnis passen
% komplett den Richtlinien entsprechend wäre irgendwas in Richtung 80-85pt
\setlength{\cftfignumwidth}{90pt}
\renewcommand{\cftfigpresnum}{\bfseries{}Abbildung }
\renewcommand{\cftfigaftersnum}{\normalfont}

% TODO: das gleiche Spiel für das Tabellenverzeichnis

% Abkürzungsverzeichnis
\renewcommand{\nomname}{Abkürzungsverzeichnis}
\setlength{\nomitemsep}{-\parsep}

\definecolor{brown}{RGB}{204,102,0}
\definecolor{darkgreen}{RGB}{0,102,0}

\lstdefinestyle{coloredpython}{
    breaklines=true,
    language=Python,
    showspaces=false,
    showstringspaces=false,
    basicstyle=\footnotesize,
    commentstyle=\color{darkgreen},
    keywordstyle=\color{blue},
    stringstyle=\color{brown},
}

\lstdefinestyle{coloredc}{
    breaklines=true,
    language=C,
    showspaces=false,
    showstringspaces=false,
    basicstyle=\footnotesize,
    commentstyle=\color{darkgreen},
    keywordstyle=\color{blue},
    stringstyle=\color{brown},
}

% gediebt von http://tex.stackexchange.com/a/50263
\lstdefinelanguage{diff}{
    morecomment=[f][\color{blue}]{@@},
    morecomment=[f][\color{blue}]{diff},
    morecomment=[f][\color{red}]{-},
    morecomment=[f][\color{green}]{+},
    morecomment=[f][\color{magenta}]{---},
    morecomment=[f][\color{magenta}]{+++},
}

% heißt nur wegen der Konsistenz *colored*diff, denn farbig ist er auch schon durch die Defintion
\lstdefinestyle{coloreddiff}{
    breaklines=true,
    language=diff,
    showspaces=false,
    showstringspaces=false,
    basicstyle=\footnotesize,
}

\lstdefinestyle{coloredbash}{
    breaklines=true,
    language=bash,
    showspaces=false,
    showstringspaces=false,
    basicstyle=\footnotesize,
    commentstyle=\color{darkgreen},
    keywordstyle=\color{blue},
    stringstyle=\color{brown},
}


% ursprünglich für urldate genutzt, nun nicht mehr
% wird vielleicht aber irgendwann nochmal nützlich, deswegen bleibt es
\newcommand{\swallowarialspace}{\kern-0.55ex}

% Definitionen und Befehle für das Literatütenverzeichnis
\NewBibliographyString{urlin}
\DefineBibliographyStrings{german}{
    urlin = {In},
    references = {Quellenverzeichnis},
}

\DeclareFieldFormat{url}{\bibstring{urlin}\addcolon\space\url{#1}}
\DeclareFieldFormat{urldate}{(#1)}

\DeclareCiteCommand{\citeurldate}
{\boolfalse{citetracker}
\boolfalse{pagetracker}
\usebibmacro{prenote}}
{\printurldate}
{\multicitedelim}
{\usebibmacro{postnote}}

% mehrere Autoren mittels ; trennen
\renewcommand{\multinamedelim}{\addsemicolon\space}
\renewcommand{\finalnamedelim}{\addsemicolon\space}

% Zitat-Befehle, zugeschnitten auf die Richtlinien der BA Glauchau
% die Counter werden für eine Warnung vor der ausschließlichen Verwendung von
% Internetquellen genutzt
% zum Deaktivieren der Warnung kann der Counter nononlinerefs auf einen Wert
% ungleich 0 gesetzt werden
\newcounter{onlinerefs}
\newcommand{\onlinecite}[2][]{\stepcounter{onlinerefs}\footnote{online: \citeauthor{#2}, \citeyear[][#1]{#2}\swallowarialspace{}\citeurldate{#2}}}

\newcounter{nononlinerefs}
\newcommand{\indcite}[2][]{\stepcounter{nononlinerefs}\footnote{vgl. \citeauthor{#2}, \citeyear[][#1]{#2}}}

\newcommand{\litcite}[2][]{\stepcounter{nononlinerefs}\footnote{\citeauthor{#2}, \citeyear[][#1]{#2}}}

% FIXME: warum funktioniert das nicht?
%\newcommand{\seccite}[4][][]{\stepcounter{nononlinerefs}\footnote{\citeauthor{#3}, \citeyear[][#1]{#3} zit. nach \citeauthor{#4}, \citeyear[][#2]{#4}}}

\DeclareNameAlias{default}{last-first}

\DeclareBibliographyDriver{online}{
    \printnames{author}
    \addcolon\space\newblock
    \printfield{title}
    \newunit\newblock
    \printlist{publisher}
    \newunit
    \printlist{location}
    \addcomma\space
    \printfield{year}
    \addcomma\space\newblock
    \printfield{url}
    \printurldate
}

\DeclareBibliographyDriver{book}{
    \printnames{author}
    \addcolon\space\newblock
    \printfield{title}
    \newunit\newblock
    \printlist{publisher}
    \newunit
    \printlist{location}
    \addcomma\space
    \printfield{year}
}

\DeclareBibliographyDriver{report}{
    \printnames{author}
    \addcolon\space\newblock
    \printfield{title}
    \addcolon\space
    \printlist{institution}
    \newunit
    \printlist{location}
    \addcomma\space
    \printfield{year}
    \addcomma\space
    \printfield{url}
}

% diese Liste müsste man jetzt noch für Dissertationen und Co. fortführen
% wer das braucht, kann ja von oben abschreiben

