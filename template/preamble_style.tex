\fancypagestyle{report-page}{
    \fancyhf{}
    \fancyhead[R]{\rightmark}
    \renewcommand{\headrulewidth}{0.4pt}
    \fancyfoot[R]{Seite \thepage}
    \renewcommand{\footrulewidth}{0.4pt}
}

\fancypagestyle{plain}{
    \fancyhf{}
    \fancyhead[R]{\leftmark}
    \renewcommand{\headrulewidth}{0.4pt}
    \fancyfoot[R]{Seite \thepage}
    \renewcommand{\footrulewidth}{0.4pt}
}

% irgendwas mit den Fußnoten … alles viel zu lange her
\setlength{\footnotesep}{10pt}
% kein Absatzeinschub
\setlength{\parindent}{0pt}

% Standardschriftartenfamilie auf Sans Serif festlegen
\renewcommand{\familydefault}{\sfdefault}

% in den Verzeichnissen die Punkte zum Auffüllen bis zur Seitenzahl direkt
% hintereinander ausgeben (ohne echten Abstand)
\renewcommand{\cftdotsep}{0.1}
\renewcommand{\cftsecleader}{\cftdotfill{\cftdotsep}}

% Inhaltsverzeichnis richtig einrücken
\cftsetindents{section}{0em}{55pt}
\cftsetindents{subsection}{0em}{55pt}
\cftsetindents{subsubsection}{0em}{55pt}
\cftsetindents{paragraph}{0em}{55pt}
% um ganz den Richtlinien zu entsprechen, muss der Abstand zwischen Kapiteln
% genauso groß sein wie der zwischen Unterkapiteln, das sieht jedoch schrecklich
% und sehr unübersichtlich aus
% wer sich das dennoch freiwillig antun möchte, darf bei der folgenden Zeile
% gern das % entfernen
%\setlength{\cftbeforesecskip}{0pt}

% Abbildungsverzeichnis richtig einrücken und fett drucken
\setlength{\cftfigindent}{0pt}
% etwas großzügiger bemessen, damit auch Abbildungen mit zweistelligen Nummern
% noch ins Verzeichnis passen
% komplett den Richtlinien entsprechend wäre irgendwas in Richtung 80-85pt
\setlength{\cftfignumwidth}{90pt}
\renewcommand{\cftfigpresnum}{\bfseries{}Abbildung }
\renewcommand{\cftfigaftersnum}{\normalfont}

% TODO: das gleiche Spiel für das Tabellenverzeichnis

% Abkürzungsverzeichnis
\renewcommand{\nomname}{Abkürzungsverzeichnis}
\setlength{\nomitemsep}{-\parsep}

% Seitennummern aus dem Anhangverzeichnis entfernen
\cftpagenumbersoff{appendices}

%all about codeboxes
\definecolor{brown}{RGB}{204,102,0}
\definecolor{darkgreen}{RGB}{0,102,0}

% gediebt von http://tex.stackexchange.com/a/50263
\lstdefinelanguage{diff}{
	morecomment=[f][\color{blue}]{@@},
	morecomment=[f][\color{blue}]{diff},
	morecomment=[f][\color{red}]{-},
	morecomment=[f][\color{green}]{+},
	morecomment=[f][\color{magenta}]{---},
	morecomment=[f][\color{magenta}]{+++},
}

\lstdefinestyle{code}{
	basicstyle=\footnotesize,	% the size of the fonts that are used for the code
	breakatwhitespace=true,		% sets if automatic breaks should only happen at whitespace
	breaklines=true,			% sets automatic line breaking
	captionpos=b,				% sets the caption-position to bottom
	keepspaces=true,			% keeps spaces in text, useful for keeping indentation of code (possibly needs columns=flexible)
	showspaces=false,			% show spaces everywhere adding particular underscores; it overrides 'showstringspaces'
	showstringspaces=false,		% underline spaces within strings only
	tabsize=4,					% sets default tabsize to 2 spaces	
	title=\lstname				% show the filename of files included with \lstinputlisting; also try caption instead of title
}

\lstdefinestyle{numbered}{
	numbers=left,							% where to put the line-numbers; possible values are (none, left, right)
	numbersep=6pt,							% how far the line-numbers are from the code
	numberstyle=\footnotesize\color{gray},	% the style that is used for the line-numbers
    stepnumber=1,							% the step between two line-numbers. If it's 1, each line will be numbered
}

\lstdefinestyle{colored}{
	commentstyle=\color{darkgreen},
	keywordstyle=\color{blue},
	stringstyle=\color{brown},
	style=code,
}
\lstdefinestyle{default}{
	style=colored,
	style=code,
}

\lstdefinestyle{coloredpython}{	%deprecated, better use {language=python,style=default}
	language=Python,
	style=colored,
	style=code,
}

\lstdefinestyle{coloredc}{	%deprecated
	language=C,
	style=colored,
	style=code,
}

\lstdefinestyle{coloredbash}{	%deprecated
	language=bash,
	style=colored,
	style=code,
}

% heißt nur wegen der Konsistenz *colored*diff, denn farbig ist er auch schon durch die Defintion
\lstdefinestyle{coloreddiff}{
    language=diff,
    style=code,
}
              





