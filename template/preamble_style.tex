\fancypagestyle{report-page}{
    \fancyhf{}
    \fancyhead[R]{\rightmark}
    \renewcommand{\headrulewidth}{0.4pt}
    \fancyfoot[R]{Seite \thepage}
    \renewcommand{\footrulewidth}{0.4pt}
}

\fancypagestyle{plain}{
    \fancyhf{}
    \fancyhead[R]{\leftmark}
    \renewcommand{\headrulewidth}{0.4pt}
    \fancyfoot[R]{Seite \thepage}
    \renewcommand{\footrulewidth}{0.4pt}
}

% irgendwas mit den Fußnoten … alles viel zu lange her
\setlength{\footnotesep}{10pt}
% kein Absatzeinschub
\setlength{\parindent}{0pt}

% Standardschriftartenfamilie auf Sans Serif festlegen
\renewcommand{\familydefault}{\sfdefault}

% in den Verzeichnissen die Punkte zum Auffüllen bis zur Seitenzahl direkt
% hintereinander ausgeben (ohne echten Abstand)
\renewcommand{\cftdotsep}{0.1}
\renewcommand{\cftsecleader}{\cftdotfill{\cftdotsep}}

% Inhaltsverzeichnis richtig einrücken
\cftsetindents{section}{0em}{55pt}
\cftsetindents{subsection}{0em}{55pt}
\cftsetindents{subsubsection}{0em}{55pt}
\cftsetindents{paragraph}{0em}{55pt}
% um ganz den Richtlinien zu entsprechen, muss der Abstand zwischen Kapiteln
% genauso groß sein wie der zwischen Unterkapiteln, das sieht jedoch schrecklich
% und sehr unübersichtlich aus
% wer sich das dennoch freiwillig antun möchte, darf bei der folgenden Zeile
% gern das % entfernen
%\setlength{\cftbeforesecskip}{0pt}

% Abbildungsverzeichnis richtig einrücken und fett drucken
\setlength{\cftfigindent}{0pt}
% etwas großzügiger bemessen, damit auch Abbildungen mit zweistelligen Nummern
% noch ins Verzeichnis passen
% komplett den Richtlinien entsprechend wäre irgendwas in Richtung 80-85pt
\setlength{\cftfignumwidth}{90pt}
\renewcommand{\cftfigpresnum}{\bfseries{}Abbildung }
\renewcommand{\cftfigaftersnum}{\normalfont}

% TODO: das gleiche Spiel für das Tabellenverzeichnis

% Abkürzungsverzeichnis
\renewcommand{\nomname}{Abkürzungsverzeichnis}
\setlength{\nomitemsep}{-\parsep}

\definecolor{brown}{RGB}{204,102,0}
\definecolor{darkgreen}{RGB}{0,102,0}

\lstdefinestyle{coloredpython}{
    breaklines=true,
    language=Python,
    showspaces=false,
    showstringspaces=false,
    basicstyle=\footnotesize,
    commentstyle=\color{darkgreen},
    keywordstyle=\color{blue},
    stringstyle=\color{brown},
}

\lstdefinestyle{coloredc}{
    breaklines=true,
    language=C,
    showspaces=false,
    showstringspaces=false,
    basicstyle=\footnotesize,
    commentstyle=\color{darkgreen},
    keywordstyle=\color{blue},
    stringstyle=\color{brown},
}

% gediebt von http://tex.stackexchange.com/a/50263
\lstdefinelanguage{diff}{
    morecomment=[f][\color{blue}]{@@},
    morecomment=[f][\color{blue}]{diff},
    morecomment=[f][\color{red}]{-},
    morecomment=[f][\color{green}]{+},
    morecomment=[f][\color{magenta}]{---},
    morecomment=[f][\color{magenta}]{+++},
}

% heißt nur wegen der Konsistenz *colored*diff, denn farbig ist er auch schon durch die Defintion
\lstdefinestyle{coloreddiff}{
    breaklines=true,
    language=diff,
    showspaces=false,
    showstringspaces=false,
    basicstyle=\footnotesize,
}
