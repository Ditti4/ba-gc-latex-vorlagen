\NewBibliographyString{urlin}
\DefineBibliographyStrings{german}{
    urlin = {In},
    references = {Quellenverzeichnis},
}

\DeclareFieldFormat{url}{\bibstring{urlin}\addcolon\space\url{#1}}
\DeclareFieldFormat{urldate}{(#1)}

\DeclareCiteCommand{\citeurldate}
{\boolfalse{citetracker}
\boolfalse{pagetracker}
\usebibmacro{prenote}}
{\printurldate}
{\multicitedelim}
{\usebibmacro{postnote}}

% mehrere Autoren mittels ; trennen
\renewcommand{\multinamedelim}{\addsemicolon\space}
\renewcommand{\finalnamedelim}{\addsemicolon\space}

% Zitat-Befehle, zugeschnitten auf die Richtlinien der BA Glauchau
% die Counter werden für eine Warnung vor der ausschließlichen Verwendung von
% Internetquellen genutzt
% zum Deaktivieren der Warnung kann der Counter nononlinerefs auf einen Wert
% ungleich 0 gesetzt werden
\newcounter{onlinerefs}
\newcommand{\onlinecite}[2][]{\stepcounter{onlinerefs}\footnote{online: \citeauthor{#2}, \citeyear[][#1]{#2}\swallowarialspace{}\citeurldate{#2}}}
% \onlinesource funktioniert ähnlich wie \onlinecite, jedoch wird mit source keine Fußnote angelegt, sondern die Quellenangabe wird direkt ausgeschrieben
% das gleiche gilt für die übrigen *source-Befehle
\newcommand{\onlinesource}[2][]{\stepcounter{onlinerefs}online: \citeauthor{#2}, \citeyear[][#1]{#2}\swallowarialspace{}\citeurldate{#2}}
\newcommand{\indonlinecite}[2][]{\stepcounter{onlinerefs}\footnote{vgl. online: \citeauthor{#2}, \citeyear[][#1]{#2}\swallowarialspace{}\citeurldate{#2}}}
\newcommand{\indonlinesource}[2][]{\stepcounter{onlinerefs}vgl. online: \citeauthor{#2}, \citeyear[][#1]{#2}\swallowarialspace{}\citeurldate{#2}}

\newcounter{nononlinerefs}
\newcommand{\indcite}[2][]{\stepcounter{nononlinerefs}\footnote{vgl. \citeauthor{#2}, \citeyear[][#1]{#2}}}
\newcommand{\indsource}[2][]{\stepcounter{nononlinerefs}vgl. \citeauthor{#2}, \citeyear[][#1]{#2}}

\newcommand{\litcite}[2][]{\stepcounter{nononlinerefs}\footnote{\citeauthor{#2}, \citeyear[][#1]{#2}}}
\newcommand{\litsource}[2][]{\stepcounter{nononlinerefs}\citeauthor{#2}, \citeyear[][#1]{#2}}

% FIXME: warum funktioniert das nicht?
%\newcommand{\seccite}[4][][]{\stepcounter{nononlinerefs}\footnote{\citeauthor{#3}, \citeyear[][#1]{#3} zit. nach \citeauthor{#4}, \citeyear[][#2]{#4}}}

\DeclareNameAlias{default}{last-first}

\DeclareBibliographyDriver{online}{
    \printnames{author}
    \addcolon\space\newblock
    \printfield{title}
    \newunit\newblock
    \printlist{publisher}
    \newunit
    \printlist{location}
    \addcomma\space
    \printfield{year}
    \addcomma\space\newblock
    \printfield{url}
    \printurldate
}

\DeclareBibliographyDriver{book}{
    \printnames{author}
    \addcolon\space\newblock
    \printfield{title}
    \newunit\newblock
    \printlist{publisher}
    \newunit
    \printlist{location}
    \addcomma\space
    \printfield{year}
}

\DeclareBibliographyDriver{report}{
    \printnames{author}
    \addcolon\space\newblock
    \printfield{title}
    \addcolon\space
    \printlist{institution}
    \newunit
    \printlist{location}
    \addcomma\space
    \printfield{year}
    \addcomma\space
    \printfield{url}
}

% diese Liste müsste man jetzt noch für Dissertationen und Co. fortführen
% wer das braucht, kann ja von oben abschreiben
